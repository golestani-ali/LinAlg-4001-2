\documentclass{beamer}
\usepackage{xcolor}
\usepackage{tikz}
\usepackage{adjustbox}
\usepackage{amsmath}
\usepackage{amssymb}
\usepackage{amstext}
\usepackage{amsfonts}
\usepackage{pifont}
\usepackage{mathtools}
\usepackage{cite}

\newcommand{\Lagr}{\mathcal{L}}
\newcommand{\Algebra}{\mathcal{A}}
\newcommand{\power}{\mathcal{P}}

\setbeamertemplate{headline}{%
    \leavevmode%
    \hbox{%
        \begin{beamercolorbox}[wd=\paperwidth,ht=3ex,dp=1.125ex]{palette tertiary}%
        \insertsectionnavigationhorizontal{\paperwidth}{}{}
        \end{beamercolorbox}%
    }
}

\usetikzlibrary{mindmap}
\usetheme{Madrid}
\usecolortheme{beaver}

% title page
\title[Linear Algebra 101] %optional
{Introduction into linear algebra}

\subtitle{Mathematical Preliminaries}

% \author*[1,2]{\fnm{Ali} \sur{Golestani}}\email{golestani\_ali@mathdep.iust.ac.ir}
% \author[1,2]{\fname{Ali}} % (optional, for multiple authors)
\author[Golestani, Ali]{Ali Golestani}
% \institute[IUST]{\inst{1} Iran University of Science and Technology \and %
%                       \inst{2} Iran University of Science and Technology \and
%                       \inst{3} Iran University of Science and Technology}
% % {Ali Golestani\inst{}}

\institute[IUST] % (optional)
{
  \inst{}%
  Department of Mathematics\\
  Iran University of Science and Technology
}

\date[IUST 2023] % (optional)
{Iran University of Science and Technology, Feb 2023}

\logo{\includegraphics[height=1.5cm]{"University Logo.png"}}


\begin{document}

\frame{\titlepage}

\begin{frame}
    \frametitle{Mathematical Preliminaries}
\framesubtitle{Fields}

\begin{block}{Definition 1.1}
    Let $S$ be a non-empty set and $\phi : S \times S \rightarrow S$ a function.\\
    This function is called a \underline{Binary Operator} and for the image of $(a,b) \in S$ under the function $\phi$ we write $\phi (a,b)$
    or simply $a \cdot b$.
\end{block}

\onslide<2->
\begin{block}{Definition 1.2}
    Let $S$ be a non-empty set and $\phi$ a binary operator. The orbit of $x \in S$ is defined as follows:\\
    $orb(x) = \{x^j; j \in \mathbb{N}\}$ 
\end{block}

\end{frame}

\begin{frame}
    \frametitle{Fields}

    \begin{alertblock}{Corollary 1.1}
        Note that the set $S$ forms clouser under the operation $\phi$.\\
        Meaning, for every $a,b \in S$ the product $a \cdot b \in S$
    \end{alertblock}
\end{frame}



\begin{frame}
    \frametitle{Fields}

\begin{block}{Definition 1.3}
    Let $S$ be a non-empty set and $\phi$ and $\psi$ two binary operations on $S$. The triplet $(S, \phi, \psi)$ is called a field if the following
    conditions are satisfied:
\end{block}

    \onslide<2->
\begin{block}{Conditions on $\phi$}
    \begin{itemize}
        \item <3-> $\phi$ is commutative.
        \item <4-> $\phi$ is associative.
        \item <5-> There exists a \underline{unique} element $o \in S$ s.t. $\phi(x,o) = x$ for all $x \in S$.
        \item <6-> For every $x \in S$ there exists a \underline{unique} element $-x \in S$ s.t. $\phi(x,-x) = o$.
    \end{itemize}
\end{block}
\end{frame}

\begin{frame}
    \frametitle{Fields}
\begin{block}{Conditions on $\psi$}
    \begin{itemize}
        \item $\psi$ is commutative.
        \item <2->$\psi$ is associative.
        \item <3->There exists a \underline{unique} non-zero element $e \in S$ s.t. $\psi(x,e) = x$ for all $x \in S$.
        \item <4->For every non-zero element $x \in S$ there exists a \underline{unique} element $x^{-1} \in S$ s.t. $\psi(x,x^{-1}) = e$.
        \item <5->$\psi$ distributes over $\phi$.
    \end{itemize}
\end{block}
\end{frame}

\begin{frame}
    \frametitle{Fields}
Note: From now on, we take $\phi$ to be addition and $\psi$ to be multiplication; we also show them with the symbols $(+)$ and $(\cdot)$.\\
Also, we show zero element $o$ as $0$ and identity element $e$ as $1$ if needed.
\end{frame}

\begin{frame}
    \frametitle{Fields}

    \begin{exampleblock}{example 1.1}
        The set of complex numbers $\mathbb{C}$ forms a field under the addition and multiplication of complex numbers.
    \end{exampleblock}

    \onslide<2->
    \begin{exampleblock}{example 1.2}
        The set of rational numbers $\mathbb{Q}$ forms a field under the ordinary addition and multiplication.
    \end{exampleblock}

    \onslide<3->
    \begin{exampleblock}{example 1.3}
        The set $\mathbb{C}[\sqrt{2}]$ forms a field under the addition and multiplication of complex numbers. (Assignment)
    \end{exampleblock}

\end{frame}

\begin{frame}
    \frametitle{Vector Spaces}
    
    \begin{block}{Definition 1.4}
        Let $K$ be a field and $V$ be a non-empty set of objects (aka vectors). The tuple $(V,K)$ is called a linear space iff:

        \onslide<2->
        \begin{itemize}
            \item V consists of an addition operator $(+)$ satisfying the first four conditions of Addition (Definition 1.3).
            \item For every $k,k' \in K$ and $v,v_1,v_2 \in V$:
            \begin{itemize}
                \item $k(k'v) = kk'(v)$
                \item $k(v_1 + v_2) = kv_1 + kv_2$
            \end{itemize}
        \end{itemize}
    \end{block}
\end{frame}

\begin{frame}
    \frametitle{Vector Spaces}

    \begin{exampleblock}{example 1.4}
        Let $K$ be a field, and $V$ be the set of all n-tuples $(x_1,x_2, \dots, x_n)$ consisting of elements in $K$.\\
        Define multiplication and addition as below:
        \begin{itemize}
            \item $(+)$: For every $(x_i)_{i=1}^{n}$ and $(x_j)_{j=1}^{n}$, $(x_i)+(x_j)= (x_i+x_j)$.
            \item $(\cdot)$: For every $k \in K$, $k(x_i)_{i=1}^{n} = (kx_i)_{i=1}^{n}$
        \end{itemize}
        The tuple $(V,K)$ is a linear space.
    \end{exampleblock}
\end{frame}

\end{document}