\documentclass{beamer}
\usepackage{xcolor}
\usepackage{tikz}
\usepackage{adjustbox}
\usepackage{amsmath}
\usepackage{amssymb}
\usepackage{amstext}
\usepackage{amsfonts}
\usepackage{pifont}
\usepackage{mathtools}
\usepackage{cite}

\newcommand{\Lagr}{\mathcal{L}}
\newcommand{\Algebra}{\mathcal{A}}
\newcommand{\power}{\mathcal{P}}

\setbeamertemplate{headline}{%
    \leavevmode%
    \hbox{%
        \begin{beamercolorbox}[wd=\paperwidth,ht=3ex,dp=1.125ex]{palette tertiary}%
        \insertsectionnavigationhorizontal{\paperwidth}{}{}
        \end{beamercolorbox}%
    }
}

\usetikzlibrary{mindmap}
\usetheme{Madrid}
\usecolortheme{beaver}

% title page
\title[Linear Algebra 101] %optional
{Introduction into linear algebra}

\subtitle{Mathematical Preliminaries}

% \author*[1,2]{\fnm{Ali} \sur{Golestani}}\email{golestani\_ali@mathdep.iust.ac.ir}
% \author[1,2]{\fname{Ali}} % (optional, for multiple authors)
\author[Golestani, Ali]{Ali Golestani}
% \institute[IUST]{\inst{1} Iran University of Science and Technology \and %
%                       \inst{2} Iran University of Science and Technology \and
%                       \inst{3} Iran University of Science and Technology}
% % {Ali Golestani\inst{}}

\institute[IUST] % (optional)
{
  \inst{}%
  Department of Mathematics\\
  Iran University of Science and Technology
}

\date[IUST 2023] % (optional)
{Iran University of Science and Technology, Feb 2023}

\logo{\includegraphics[height=1.5cm]{"University Logo.png"}}


\begin{document}

\frame{\titlepage}

\begin{frame}
    \frametitle{Mathematical Preliminaries}
    \framesubtitle{Set Theoretic Concepts}
    
\begin{block}{Definition 1.1}
    Let $X$ and $Y$ be two non-empty sets and $A$ and $B$ subsets of these sets respectively; If there exists a relation  $R: A \rightarrow B$ s.t.
    for every $a \in A$ there exists only one $b \in B$ s.t. $R(a) = b$, then $R$ is a function from $X$ to $Y$.
\end{block}

\end{frame}

\begin{frame}
    \frametitle{<title>}

    \begin{block}{Definition 1.2}
        Let $X$ and $Y$ be two non-empty sets and $f: X \rightarrow Y$ be a function; If 
        \begin{itemize}
            \item For every $y \in Y$ there exists $x \in X$ s.t. $f(x) = y$
        \end{itemize}
    \end{block}

\end{frame}

\end{document}